\section{Introduction}

This documents details the procedure to maintain software versions updated and operating system security patches. 

The scope of this document applies to all IT computer and network devices installed in Rubin's network

While there are many responses to vulnerabilities that could have potentially the same effect of mitigating the risk, this document will only consider the patching process. 


\section{Definitions}

\begin{itemize}
    \item Severity: Impact of the vulnerability that the patch is suppose to correct. 
    \item Patching Started: Time when the patching process starts. It should always start in the development environment.
    \item Patching Completed: Time when the patching process has completed. This is when production environment has been patched against the vulnerability.
\end{itemize}

\section{Vulnerability Management}

\subsection{Acknowledge of the Vulnerability}

TBD

\subsection{Plan Response}

The response to the vulnerability will be planned according to its severity.

The scheduling of the response will be communicated during the CAP and Summit Coordinations meetings. 


\subsubsection{Patch Priority}

Priority of the patch will be based on the severity of the vulnerability.
Severity ratings are based on the \href{https://nvd.nist.gov/vuln-metrics/cvss}{Common Vulnerability Scoring System}.  

\begin{center}
    \begin{tabular}{ |>{\columncolor[gray]{0.8}}p{3cm}||p{5cm}|p{5cm}|  }
        \hline
        \rowcolor{lightgray}
        Severity& Patching Started & Patching Completed\\
        \hline
        Critical   & Within 48 hours of patch release    & Within 1 week of patch release\\
        \hline
        High&   Within 72 hours of patch release  & Within 2 weeks of patch release\\
        \hline
        Medium & Within 1 week of patch release & Within 3 weeks of patch release\\
        \hline
        Low    & Within 1 month of patch release & Within 2 months of patch release\\
        \hline
       \end{tabular}
\end{center}


\subsection{Execute Response}

The execution of the response will follow this sequence

\begin{itemize}
    \item Deployment in Development Environemt (TTS)
    \item Schedule Deployment in Testing Environemt (BTS)
    \item Schedule Deployment in Production Environemt
\end{itemize}

However, under very particular conditions related to the severity and impact of the vulnerability, the sequence may be modified. 

\section{Exceptions}

All requests for exceptions to patching must be submitted to the DevOps team using the Jira ticketing system. Requests must be informed at least 2 cycles before its corresponding patching. 

Systems under the exception list will receive additional network controls to mitiage risks. 